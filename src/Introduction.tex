%! Author = alexis
%! Date = 14/06/2022

% Document


\section{Introduction}

Social scientists such as historians and sociologists want to make sense of the structure and dynamics of the social relationships between people of a given place and time.
Social Network Analysis (SNA) and its history equivalent Historical Social Network Analysis (HSNA) is one of the main paradigm to achieve this task.
It consists in constructing a network representing the social ties between the persons of interest, and studying this network to make sociological conclusions. Usually, persons are represented as nodes in the network, while the links model social relationships, such as friendships or family links. For this, social scientists try to exhaustively list all the persons in a restricted time and place with all their social ties, and create a network from it.
The resulting network is considered to be a good model of the social reality, thus allowing us to study the structure and dynamics of the social fabric of a period, by studying the network in itself.
In parallel, a lot of work has been done in network visualization and specifically Social Network Visualization (SNV) to make useful representations of social networks, and visual analytics tools allowing an effective exploration and analysis of this type of data.
However, sociology and history data can be quite complex, and simple networks model are often a simplification of the real world social phenomena.
Several network models ranging from simple to complex ones have been introduced, with associated visual representations and tools. But there is no consensus of which is better, specifically for networks constructed from historical sources, and the majority of research is still done using very simple models, with a classical node-link representation.
Furthermore, current SNA tools such as Gephi or Pajek do not provide much guidance to the social scientists for their analysis, which often require a good computer science and statistics background.
This thesis aim is to tackle those two problems, by first defining a network model which models well most of the historical sources we encountered, and proposing visual representations to explore them.
Then, we propose visual analytics tools and methods specifically designed for social scientists to explore their data, with the aim of proposing the right balance between algorithmic power, interpretation of the analysis and the decision.

\subsection{Social History and Historical Social Network Analysis}

History is the science of retrieving and characterizing facts about the past, in all their complexity. Traditional history methodology consists in finding and expliciting specific events---such as wars or diplomatic tensions---and eliciting their causes and consequences, and narrating the lives of historic figures, such as reigners or artists. But in the first half of the 20th century, a new history approach and methodology emerged called social history. This branch of history studies the socio-economic dynamics between the different groups of a society, instead of focusing on the affairs of a state.
More recently, wth the development of network science and computer science, sociologists started to study social phenomena and relationships from a network perspective. A network is an abstraction used to modelize phenomena based on relationships between entities, made of nodes and links. By modeling the social ties of a group of interest, such as preschooler [REF], or a karate club [REF] with the use of a network, sociologists can leverage quantitative measures from the network to make sociological conclusions.
This network analysis approach grew in popularity in recent years, and has started to be used and formalized by historians, under the term of Historical Network Research (HNR). Similarly to sociologists, historians can build a network modelizing the social relationships of actors of the past, restricted in a specific period and area they are studying. If sociologists can use surveys, experiments or nowadays the internet to extract social relationships and construct a social network, historians are restricted by the written sources they can find. Their main source of work to extract social relationships in a rigorous way are historical documents which correspond to traces of specific events linking people together. These can be marriage acts, birth certificates, or census for family and close personal relationships, or migration acts and working contracts for other types of social ties. After having a selected corpus, they have to annotate manually each document to extract the persons mentioned in it along the relationships between them, to finally construct a network from this data. This is a long and tedious process which can result in small to large networks that they want to analyze to make conclusions on the social dynamics of a population of interest.
For this complicated task, historians follow what is called a Social Network Analysis (SNA), or more precisely a HIstorical Social Network Analysis (HSNA) which consists in characterizing the structure of the network with measures such as the centrality or the density of some parts of the network to then make conclusions on how people were interacting in the period of interest. To help their analysis, and generate new hypotheses, they usually rely on Visual Analytics tools to represent and explore their network. The elaboration of visual tools to represent and explore social networks is called Social Network Visualization.
Sociologists and historians started to use static representation of networks, using node-links diagrams to have a visual understanding of their data, and to report their findings in publications. With the development of visualization, more complex representations and visual analytics tools emerged, which allow more complex representation and exploration capabilities, with the help of interactions and navigations features. Social networks visual systems such as Gephi or Pajek are now widely used in HNR and SNA by social scientists. Representing their network data and being able to interact with it allows them to rapidly have an overview of it, confirm hypotheses they have and arrange new ones by exploring the network.

However, most used social networks visual analysis tools still have several issues that we tackle in this thesis : the visual representations still widely used are pretty simple, and are often not a good fit to represent and explore complex multivariate historical dataset, and current visual analytics tools often do not provide enough power and guidance to the end users to manipulate their data, which can result in frustration.

\subsection{Network models and representations}

Person-to-person simple node-link diagram is still the most widely used network representation is SNA, and most SVA tools only include this type of representation. This visualization shows the persons as nodes, and social ties as links and displays them in a way to minimize the number of crossings to increase the readability. However, historians very often have access to richer and more diverse information through the historical documents they study. The documents can refer to coexistent complex social relationships which link several people together with different roles. These cannot be modeled with simple person-to-person links, without losing some information on the social implication of these relationships. Moreover, documents often give access to other information related to the event they refer to, such as the time, the location or the roles of the different persons mentioned. For example, marriage acts often indicate the date and the place of the event, and mention persons under different roles : the spouses, the witness, the parents, the priest etc. Additional information related to persons can also be mentioned, such as their age, origin or profession. It is clear that simply using a person network model won’t encapsulate the whole complexity of the data and will simplify the social relationship. This is a common issue in SNA and HSNA [REF Lemercier] and more complex network models are needed. However, complexity
along with visual analysis tools to explore them.

\subsection{Usability Issues}

One of the aims of Visual Analytics is to provide automatic or semi-automatic processing and analysis tools with data mining and machine learning algorithms, to help end users make sense of their data and find interesting patterns and relationships.
However, current social network visual analytics systems are still very algorithm oriented, and do not provide many controls to historians and sociologists who usually feel off the analysis loop when the system provides automatic and algorithmic results. One of the reason is because automatic results can be hard to interpret, especially in a discipline such as History or Sociology, where users often have little knowledge on computer science.
One example is the automatic detection of community structures using network clustering algorithms. Social networks are known to have a community-like structure, meaning that the probability of a link existing between two random person nodes is not uniform, and that people tend to agglomerate in groups, who have more social ties between them than with other persons in the network.
There are a lot of existing clustering algorithms which aim to automatically find these groups, by optimizing measures such as the modularity or using propagation models.
However, clustering is an ill-defined problem, and several good partitions may coexist for the same network, and which can have several interpretations in a SNA. Most SNA/SVA tools such as Gephi or NodeXl provide several well known clustering algorithms such as Girvan-Newman, Louvain or Clauset-Newman-Moore, but do not provide much guidance on how to use them and interpret their results.
Social Scientists often try several ones in the list of algorithms proposed until finding a convenient result, in the eyes of the analysis they want to follow.
This leads to a non satisfactory analysis process, as historians are out of the loop and have few decisions on the results. This usability issue is the same for automatic processes with no universal ground truth.

\subsection{Contribution and research statement}

This thesis is centered around two research questions: first, the proposition of an efficient network model to represent historical sources as a network, with associated visualizations to show and explore this type of network. Secondly, elaborating visual analytics tools to explore this type of data with the right balance of algorithmic power, simplicity and interpretability for the social scientists, who need to be in control of the analysis. We first [tell plan]
