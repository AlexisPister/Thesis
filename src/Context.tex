%! Author = alexis
%! Date = 20/06/2022

\section{Historical Network Analysis and Visualization}

\subsection{Social History}

\subsubsection{Social History goals}

Historians try to understand an epoch using textual sources from the past, and trying to extract useful information from them.
Social history, which is a branch of history focus on understanding how societies were organised and how people were living together at a particular time and place. Charles Tilly argued that the task of social history lays in "(i) documenting large structural changes, (2) reconstructing the experiences of ordinary people in the course of those changes, and (3) connecting the two". For the latter, historians can leverage personal written sources---such as letters, journals, books, and newspapers---to have the internal point of vue of persons living in this society and desriptions of lives of precise individuals.
For the former, historians usually need to study more structured documents which contain information which can be extracted in a rigorous and exhaustive way. These documents can for example be census, migration acts or marriage acts. By studying theses documents and by systematically extrating the information of these documents, historians can make global and quantitative conclusions on certain social and behavioural aspects of society of intersest.
For example .. [EXAMPLE CHANGEMENT METIERS XXth century]

\subsubsection{Historical Network Research}



\subsection{Social Network Analysis}




\subsection{Social Network Visualization}




