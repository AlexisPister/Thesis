%! Author = alexis
%! Date = 20/06/2022

\chapter{Historical Network Analysis and Visualization}

%\section{Social History}
\section{Quantititative History}

\subsection{Social History}

Historians try to understand an epoch using textual sources from the past, and trying to extract useful information from them.
Social history, which is a branch of history, focus on understanding how societies were organised and how people were living together at a particular time and place. Charles Tilly argued that the task of social history lays in "(i) documenting large structural changes, (2) reconstructing the experiences of ordinary people in the course of those changes, and (3) connecting the two". For the latter, historians can leverage personal written sources---such as letters, journals, books, and newspapers---to have the internal point of view of persons living in this society and descriptions of lives of precise individuals.
For the former, historians usually need to study more structured documents which contain information which can be extracted in a predefined and exhaustive way. These documents can for example be census, migration acts or marriage acts. By studying theses documents and by systematically extracting the information of these documents, historians can make global and quantitative conclusions on certain social and behavioural aspects of societies of interest.
For example .. [EXAMPLE CHANGEMENT METIERS XXth century]

\subsection{Historical Network Research and Network Modeling}

History started to adopt some of the methods and vocabulary of Network research in the 1980s, several years after other fields such as Sociology or Anthropology (TO CHECK). Before that, historians were already describing relational atructures when studying family and organization. It was often a part of discussion and a conclusion of several studies. Network reseach was a way to put these relational strcutures as an object of study in iteself, and allowed to study them in a more systematic and quantitative way.
Instead of only looking at classes and groups, historians thus started to look at relational links before individual, such as family, friendships or businedd ties.
They already had techniques and tools to annotate and extract quantittative information from textual sources that they adapted to study relational links, from textual sources.
We then could see the emergence of study based on HNR, where historians constructed networks from their sources, to follow HSNA and make conclusions on the ties .
However, constructing a network from historical sources, which can differ in their structure is not a trivial task. The most straightforward approach, based on the most well known social network analysis, consists in constructing social network based on simple graph $G = (V, E)$ with $V$ a set a vertices representing the persons of interest, and $E \subseteq V^2$ a set of edges modeling the social ties between pairs of persons. This allows to have a simple network to visualize and analyze, but do not always reflect the social complexity of the real relationships. More complex networks models have been proposed in SNA to be able to model more complex social relationships.



\section{Social Network Analysis}




\section{Social Network Visualization}




