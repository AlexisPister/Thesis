\newcommand{\abstracteng}{
%    \rev{This thesis aims at identifying how visual analytics can support historians in their social network analysis process, from the collection of historical documents to the formulation of high-level socio-historical conclusions.}
    \rev{This thesis aims at identifying how visual analytics can support historians in their social network analysis process.}
    \rev{Historical social network analysis is a method to study social relationships between groups of actors (families, institutions, companies, etc.) to understand their underlying structure while characterizing specific behaviors.
    Social historians are able to reconstruct relationships of the past using historical documents' content, such as marriage acts, migration forms, birth certificates, and censuses.}
    Through visualization and analytical methods, they can describe the global structure of studied groups and explain individual behaviors through local network patterns.
%    However, the inspection, encoding, correction, and modeling process of the historical documents leading to a finalized network is intricate and often results in inconsistencies, errors, distortions, simplifications, and traceability issues.
    \rev{However, the inspection and encoding of the sources leading to a finalized network is intricate and often results in inconsistencies, errors, distortions, simplifications, and traceability issues.}
%    \rev{Moreover, several usability and analytical interpretation problems currently limit the usage of visual interfaces in History.}
%    \rev{Moreover, usability and analytical interpretation issues often limit the usage of visual interfaces in History.}
    For these reasons, social historians are not always able to make thorough historical conclusions with current analytical and visualization tools, especially given recurrent usability and interpretability issues.
%    I aim in this thesis to identify how visual analytics---\rev{the integration of data mining capabilities into visual interfaces through interaction}---can support and guide social historians in their process, from the collection of their data to the answer to high-level historical questions.
    I aim in this thesis to identify how visual analytics---\rev{the integration of data mining capabilities into visual interfaces through interaction}---can support and guide social historians in the different steps of their process.
%    Visual analytics---\rev{the integration of data mining capabilities into visual interfaces through interaction}---can support social historians in their process, from the collection of their data to the answer to high-level historical questions.
    Towards this goal, I formalize the workflow of historical network analysis \rev{from collaborations with social historians, starting at the acquisition of sources to the final visual analysis.}
    By highlighting recurring pitfalls, I \rev{point out that tools supporting this process should satisfy traceability, simplicity, and document reality principles to ease bask and forth between the different steps, provide tools easy to manipulate, and not distort the content of sources with modifications and simplifications.
    The network modeling influence deeply those properties given the high diversity in properties of network models.
    I propose to model historical sources into bipartite multivariate dynamic social networks with roles as they provide a good tradeoff of simplicity and expressiveness while modeling explicitly the documents, hence letting users encode, correct, and analyze their data with the same abstraction and tools.}
%    Particularly, I propose to model historical sources into bipartite multivariate dynamic social networks with roles to satisfy those properties.
%Towards this goal, I first formalize the workflow of historical network analysis in collaboration with social historians, from the acquisition of their sources to their final visual analysis, and propose to model historical sources into bipartite multivariate dynamic social networks with roles to satisfy traceability, simplicity, and document reality properties.
%    This modeling allows a concrete representation of historical documents, hence letting users encode, correct, and analyze their data with the same abstraction and tools.
    I propose two interactive visual interfaces to manipulate, explore, and analyze this \rev{data model}, with a focus on usability and interpretability.
    \rev{The first system ComBiNet} allows an interactive exploration leveraging the structure, time, localization, and attributes of the data model with the help of coordinated views, a visual query system, and comparison mechanisms.
    Finding specific patterns easily and comparing them, social historians are able to find inconsistencies in their annotations and answer high-level questions.
    The second system, PK-Clustering, is a concrete proposition to increase the usability and effectiveness of clustering mechanisms in social network visual analytics systems. It consists in a mixed-initiative clustering interface that let social scientists create meaningful clusters with the help of their prior knowledge, algorithmic consensus, and interactive exploration of the network.
    Both systems have been designed with continuous feedback from social historians, and aim to increase the traceability, simplicity, and document reality of
    \rev{visual analytics supported historical social network research.}
    I conclude with discussions on the potential merging of both systems and more globally on research directions towards better integration of visual analytics systems on the whole workflow of social historians.
    Systems with a focus on \rev{those properties---traceability, simplicity, and document reality---can limit the introduction of bias} while lowering the requirements for the use of quantitative methods for historians and social scientists which has always been a controversial discussion among practitioners.
}

\newcommand{\abstractfr}{
    Cette thèse vise à identifier comment l'analyse visuelle peut aider les historiens dans leur processus d'analyse de réseaux sociaux.
    L'analyse de réseaux sociaux est une méthode utilisée en histoire sociale permettant d'étudier les relations sociales au sein de groupes d'acteurs (familles, institutions, entreprises, etc.) pour comprendre leurs structures sous-jacentes tout en décrivant des comportements individuels spécifiques.
    Les historiens peuvent reconstruire les relations du passé à partir du contenu de documents historiques, tel que des actes de mariages, des actes de naissances, ou des recensements.
%    L'utilisation de méthodes visuelles et analytiques leurs permettent alors d'explorer la strucure sociale formant ces groupes et de relier des comportements à des positions structurelles et des mesures computationelles.
    L'utilisation de méthodes visuelles et analytiques leurs permettent alors d'explorer la strucure sociale formant ces groupes et de relier des mesures structurelles à des hypothèses sociologiques et des comportements individuels.
    Cependant, l'inspection, l'encodage et la modélisation des sources pour obtenir un réseau finalisé donne souvent lieu à des erreurs, distorsions et des problèmes de traçabilité.
    Pour ces raisons, ainsi que des problèmes d'utilisabilité et d'interpretabilité, les historiens ne sont pas toujours en mesure de faire des conclusions approfondies à partir des systèmes de visualisation actuels: beaucoup d'études se limitent à une description qualitative d'images de réseaux, surlignant la présence de motifs d'intêrets (cliques, ilôts, ponts, etc.).
    Je vise dans cette thèse à identifier comment l'analyse visuelle (l'intégration d'algorithmes statistiques à des interfaces graphiques via de l'interaction) peut aider les historiens dans leur processus global.
    Dans ce but, je formalise le processus d'une analyse de réseau historique en partant de collaborations avec des historiens, de l'acquisition des sources jusqu'à l'analyse finale, en établissant que les outils aidant ce processus devraient satisfaire des principes de traçabilité, simplicité et de réalité documentaire pour faciliter les va-et-vient entre les différentes étapes, avoir des outils faciles à utiliser, et ne pas distordre le contenu des sources.
    Pour satisfaire ces propriétés, je propose de modéliser les sources historiques en réseaux sociaux bipartis multivariés dynamiques avec rôles.
    Ce modèle intègre explicitement les documents historiques sous forme de noeuds, ce qui permet à la fois aux utilisateurs d'encoder, corriger et analyser leurs données avec les mêmes outils.
    Je propose deux interfaces d'analyse visuelle pour manipuler, explorer et analyser ce modèle de données, avec un accent sur l'utilisabilité et l'interepretabilité de l'analyse.
%    Le premier système ComBiNet permet une exploration visuelle de la topologie, la dynamique, la localisation et des attributs du réseau à l'aide de vues coordonnées et d'un système de requêtes visuelles et de comparaisons.
    Le premier système ComBiNet permet une exploration visuelle de l'ensemble des dimensions du réseau à l'aide de vues coordonnées et d'un système de requêtes visuelles et de comparaisons.
    La découverte de motifs pertinents permet aux utilisateurs de découvrir des incohérences dans les annotations et de répondre à des questions socio-historiques de haut niveau.
    Le second système, PK-Clustering, constitue une proposition pour améliorer l'utilisabilité et l'efficacité des mécanismes de clustering dans les systèmes de visualisation de réseaux sociaux.
%    L'interface permet de créer des regroupements pertinents à partir de la connaissance a priori, le consensus algorithmique et l'exploration du réseau dans un cadre d'initiative mixte.
    L'interface permet de créer des regroupements pertinents à partir des connaissances a priori, du consensus algorithmique et de l'exploration du réseau dans un cadre d'initiative mixte.
    Les deux systèmes ont été conçus à partir des besoins et de retours continus d'historiens, et visent à augmenter la traçabilité, la simplicité, et la vision réelle des sources dans l'analyse de réseaux historiques.
%    Je conclus sur des discussions sur la fusion des deux systèmes et plus globalement sur la convergence vers une meilleure intégration des outils d'analyse visuelle sur le processus global des historiens.
%    Je conclus sur la possibilité de fusionner les deux systèmes et plus globalement sur la convergence vers une meilleure intégration des outils d'analyse visuelle sur le processus global des historiens.
    Je conclus sur la possibilité de fusionner les deux systèmes et plus globalement sur la necessité d'une meilleure intégration des systèmes d'analyse visuelle dans le processus de recherche des historiens.
    Cette intégration nécessite des outils plaçant les utilisateurs au centre du processus avec un accent sur l'utilisabilité et l'interpretabilité, limitant ainsi l'introduction de biais et les barrières d'utilisation des méthodes quantitatives, qui subsitent en histoire.
%    Cette intégration nécessite la prise en compte des principes de tracabilité, réalité documentaire et simplicité afin de mieux coller aux besoins et aux méthodes des historiens, et ainsi limiter l'introduction de biais et les barrières d'utilisation des méthodes quantitatives, qui subsitent en histoire.
%    la prise en compte des principes de tracabilité, réalité documentaire et simplicité afin de mieux coller aux besoins et aux méthodes des historiens, et ainsi limiter l'introduction de biais et les barrières d'utilisation des méthodes quantitatives, qui subsitent en histoire.
%    Ces outils permettraient de limiter les barrières d'utilisation des méthodes quantitatives par les historiens, qui a toujours été une discussion controversée en Histoire.
}

\newcommand{\abstractengnew}{
This thesis aims at identifying theoretically and concretely how visual analytics can support historians in their social network analysis process.
Historical social network analysis is a method to study social relationships between groups of actors (families, institutions, companies, etc.) through a reconstruction of relationships of the past from historical documents, such as marriage acts, migration forms, birth certificates, and censuses.
The use of visualization and analytical methods lets social historians explore and describe the social structure shaping those groups while explaining sociological phenomena and individual behaviours through computed network measures.
However, the inspection and encoding of the sources leading to a finalized network is intricate and often results in inconsistencies, errors, distortions, and traceability problems, and current visualization tools typically have usability and interpretability issues.
For these reasons, social historians are not always able to make thorough historical conclusions: many studies consist in qualitative description of network drawings highlighting the presence of motifs such as cliques, components, bridges, etc.
The goal of this thesis is therefore to propose visual analytics tools integrated in the global social historians workflow, with guided and easy-to-use analysis capabilities.
From collaborations with historians, I formalize the workflow of historical network analysis starting at the acquisition of sources to the final visual analysis.
By highlighting recurring pitfalls, I point out that tools supporting this process should satisfy traceability, simplicity, and document reality principles to ease bask and forth between the different steps, provide tools easy to manipulate, and not distort the content of sources with modifications and simplifications.
To satisfy those properties, I propose to model historical sources into bipartite multivariate dynamic social networks with roles as they provide a good tradeoff of simplicity and expressiveness while modeling explicitly the documents, hence letting users encode, correct, and analyze their data with the same abstraction and tools.
I then propose two interactive visual interfaces to manipulate, explore, and analyze this data model, with a focus on usability and interpretability.
The first system ComBiNet allows an interactive exploration leveraging the structure, time, localization, and attributes of the data model with the help of coordinated views and a visual query system allowing users to isolate interesting groups and individuals, and comparing their position, structures, and properties.
It also lets them highlight erroneous and inconsistent annotations directly in the interface.
The second system, PK-Clustering, is a concrete proposition to enhance the usability and effectiveness of clustering mechanisms in social network visual analytics systems. It consists in a mixed-initiative clustering interface that let social scientists create meaningful clusters with the help of their prior knowledge, algorithmic consensus, and interactive exploration of the network.
Both systems have been designed with continuous feedback from social historians, and aim to increase the traceability, simplicity, and document reality of
visual analytics supported historical social network research.
I conclude with discussions on the potential merging of both tools, and more globally on research directions towards a better integration of visual analytics systems on the whole workflow of social historians.
Systems with a focus on those properties---traceability, simplicity, and document reality---can limit the introduction of bias while lowering the requirements for the use of quantitative methods for historians and social scientists which has always been a controversial discussion among practitioners.}

%=======
%    \rev{This thesis aims at identifying theoretically and with concrete propositions how visual analytics can support historians in their social network analysis process.}
%    Historical social network analysis is a method to study social relationships between groups of actors (families, institutions, companies, etc.) through a reconstruction of relationships of the past from historical documents, such as marriage acts, migration forms, birth certificates, and censuses.
%    The use of visualization and analytical methods lets social historians explore and describe the social structure shaping those groups while explaining sociological phenomena and individual behaviors through computed network measures.
%    \rev{However, the inspection and encoding of the sources leading to a finalized network is intricate and often results in inconsistencies, errors, distortions, and traceability problems, and current visualization tools typically have usability and interpretability issues.}
%    For these reasons, social historians are not always able to make thorough historical conclusions: many studies consist of qualitative descriptions of network drawings highlighting the presence of motifs such as cliques, components, bridges, etc.
%    The goal of this thesis is therefore to propose visual analytics tools integrated into the whole social historians' workflow with guided and easy-to-use analysis capabilities.
%    Towards this goal, I formalize the workflow of historical network analysis \rev{from collaborations with social historians, starting at the acquisition of sources to the final visual analysis.}
%    By highlighting recurring pitfalls, I \rev{point out that tools supporting this process should satisfy traceability, simplicity, and document reality principles to ease bask and forth between the different steps, provide tools easy to manipulate, and not distort the content of sources with modifications and simplifications.
%    The network modeling influences deeply those properties given the high diversity in properties of network models.
%    I propose to model historical sources into bipartite multivariate dynamic social networks with roles as they provide a good tradeoff of simplicity and expressiveness while modeling explicitly the documents, hence letting users encode, correct, and analyze their data with the same abstraction and tools.}
%    I propose two interactive visual interfaces to manipulate, explore, and analyze this \rev{data model}, with a focus on usability and interpretability.
%    \rev{The first system ComBiNet} allows an interactive exploration leveraging the structure, time, localization, and attributes of the data model with the help of coordinated views, a visual query system, and comparison mechanisms.
%    Finding specific patterns easily and comparing them, social historians are able to find inconsistencies in their annotations and answer high-level questions.
%    The second system, PK-Clustering, is a concrete proposition to increase the usability and effectiveness of clustering mechanisms in social network visual analytics systems. It consists in a mixed-initiative clustering interface that let social scientists create meaningful clusters with the help of their prior knowledge, algorithmic consensus, and interactive exploration of the network.
%    Both systems have been designed with continuous feedback from social historians, and aim to increase the traceability, simplicity, and document reality of
%    \rev{visual analytics supported historical social network research.}
%    I conclude with discussions on the potential merging of both systems and more globally on research directions towards better integration of visual analytics systems on the whole workflow of social historians.
%    Systems with a focus on \rev{those properties---traceability, simplicity, and document reality---can limit the introduction of bias} while lowering the requirements for the use of quantitative methods for historians and social scientists which has always been a controversial discussion among practitioners.}
%>>>>>>> origin/overleaf-2022-10-14-2047

\newcommand{\abstractfrnew}{
Cette thèse vise à identifier théoriquement et concrètement comment l'analyse visuelle peut aider les historiens dans leur processus d'analyse de réseaux sociaux.
L'analyse de réseaux sociaux est une méthode utilisée en histoire sociale qui vise à étudier les relations sociales au sein de groupes d'acteurs (familles, institutions, entreprises, etc.) en reconstruisant les relations du passé à partir de documents historiques, tels que des actes de mariages, des actes de naissances, ou des recensements.
L'utilisation de méthodes visuelles et analytiques leurs permet d'explorer la structure sociale formant ces groupes et de relier des mesures structurelles à des hypothèses sociologiques et des comportements individuels.
Cependant, l'inspection, l'encodage et la modélisation des sources menant à un réseau finalisé donnent souvent lieu à des erreurs, des distorsions et des problèmes de traçabilité, et les systèmes de visualisation actuels présentent souvent des défauts d'utilisabilité et d'interprétabilité.
En conséquence, les historiens ne sont pas toujours en mesure de faire des conclusions approfondies à partir de ces systèmes : beaucoup d'études se limitent à une description qualitative d'images de réseaux, surlignant la présence de motifs d'intérêts (cliques, îlots, ponts, etc.).
Le but de cette thèse est donc de proposer des outils d'analyse visuelle adaptés aux historiens afin de leur permettre une meilleur intégration de leur processus global et des capacités d'analyse guidées.
En collaboration avec des historiens, je formalise le processus d'une analyse de réseau historique, de l'acquisition des sources jusqu'à l'analyse finale, en posant comme critère que les outils utilisés dans ce processus devraient satisfaire des principes de traçabilité, de simplicité et de réalité documentaire (i.e., que les données présentées doivent être conformes aux sources) pour faciliter les va-et-vient entre les différentes étapes et la prise en main par l'utilisateur et ne pas distordre le contenu des sources.
Pour satisfaire ces propriétés, je propose de modéliser les sources historiques en réseaux sociaux bipartis multivariés dynamiques avec rôles.
Ce modèle intègre explicitement les documents historiques sous forme de nœuds, ce qui permet aux utilisateurs d'encoder, de corriger et d'analyser leurs données avec les mêmes outils.
Je propose ensuite deux interfaces d'analyse visuelle permettant, avec une bonne utilisabilité et interpretabilité, de manipuler, d'explorer et d'analyser ce modèle de données.
Le premier système ComBiNet offre une exploration visuelle de l'ensemble des dimensions du réseau à l'aide de vues coordonnées et d'un système de requêtes visuelles permettant d’isoler des individus ou des groupes et de comparer leurs structures topologiques et leurs propriétés.
L’outil permet également de détecter les motifs inhabituels et ainsi de déceler les éventuelles erreurs dans les annotations.
Le second système, PK-Clustering, est une proposition d'amélioration de l'utilisabilité et de l'efficacité des mécanismes de clustering dans les systèmes de visualisation de réseaux sociaux.
L'interface permet de créer des regroupements pertinents à partir des connaissances a priori de l'utilisateur, du consensus algorithmique et de l'exploration du réseau dans un cadre d'initiative mixte.
Les deux systèmes ont été conçus à partir des besoins et retours continus d'historiens, et visent à augmenter la traçabilité, la simplicité, et la réalité documentaire des sources dans le processus d'analyse de réseaux historiques.
Je conclus sur la necessité d'une meilleure intégration des systèmes d'analyse visuelle dans le processus de recherche des historiens.
Cette intégration nécessite des outils plaçant les utilisateurs au centre du processus avec un accent sur la flexibilité et l'utilisabilité, limitant ainsi l'introduction de biais et les barrières d'utilisation des méthodes quantitatives, qui subsitent en histoire.}
