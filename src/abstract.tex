\newcommand{\abstracteng}{
Historical Social Network Analysis is a method followed by social historians to model relational phenomena of the past such as kinship, political power, migrations, or business affiliations with networks using the information of historical documents.
Through visualization and analytical methods, social historians are able to describe the global structure of such phenomena and explain individual behaviors through their network position.
However, the inspection, encoding, correction, and modeling process of the historical documents leading to a finalized network is complicated and often results in inconsistencies, errors, distortions, simplifications, and traceability issues.
For these reasons and usability issues, social historians are often not able to make thorough historical conclusions with current visualization tools. 
In this thesis, I aim to identify how visual analytics---the combination of data mining capabilities integrated into visual interfaces with direct manipulation and interaction---can support social historians in their process, from the collection of their data to the answer to high-level historical questions.
Towards this goal, I first formalize the workflow of historical network analysis in collaboration with social historians, from the acquisition of their sources to their final visual analysis, and propose to model historical sources into bipartite multivariate dynamic social networks with roles to satisfy traceability, simplicity, and document reality properties.
This modeling allows a concrete representation of historical documents, hence letting users encode, correct, and analyze their data with the same abstraction and tools.
I, therefore, propose two interactive visual interfaces to manipulate, explore, and analyze this type of data with a focus on usability for social historians.
First, I present ComBiNet, which allows an interactive exploration leveraging the structure, time, localization, and attributes of the data model with the help of coordinated views, a visual query system, and comparison mechanisms. 
Finding specific patterns easily and comparing them, social historians are able to find inconsistencies in their annotations and answer their high-level questions.
The second system, PK-Clustering, is a concrete proposition to increase the usability and effectiveness of clustering mechanisms in social network visual analytics systems. It consists in a mixed-initiative clustering interface that let social scientists create meaningful clusters with the help of their prior knowledge, algorithmic consensus, and exploration of the network.
Both systems have been designed with continuous feedback from social historians, and aim to increase the traceability, simplicity, and document reality of the historical social network analysis process.
I conclude with discussions on the potential merging of both systems and more globally on research directions towards better integration of visual analytics systems on the whole workflow of social historians.
Such systems with a focus on usability can lower the requirements for the use of quantitative methods for historians and social scientists, which has always been a controversial discussion among practitioners.
}

\abstracteng