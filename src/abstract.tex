\newcommand{\abstracteng}{
%Historical Social Network Analysis is a method followed by social historians to model relational phenomena of the past such as kinship, political power, migrations, or business affiliations with networks using the information of historical documents.
    \rev{This thesis aims at identifying how Visual Analytics can support historians in their social network analysis process, from the collection of historical documents to the formulation of high-level socio-historical conclusions.}
    \rev{Historical Social Network Analysis is a method to study social relationships between groups of actors (families, institutions, companies, etc.) to understand their underlying structure while characterizing specific behaviors.
    Social historians are able to reconstruct relationships of the past using historical documents' content, such as marriage acts, migration forms, birth certificates, and censuses.}
    Through visualization and analytical methods, they can describe the global structure of studied groups and explain individual behaviors through local network patterns.
    However, the inspection, encoding, correction, and modeling process of the historical documents leading to a finalized network is intricate and often results in inconsistencies, errors, distortions, simplifications, and traceability issues.
%    \rev{Moreover, several usability and analytical interpretation problems currently limit the usage of visual interfaces in History.}
%    \rev{Moreover, usability and analytical interpretation issues often limit the usage of visual interfaces in History.}
    For these reasons, social historians are not always able to make thorough historical conclusions with current analytical and visualization tools.
    I aim in this thesis to identify how visual analytics---\rev{the integration of data mining capabilities into visual interfaces with interaction}---can support social historians in their process, from the collection of their data to the answer to high-level historical questions.
    Towards this goal, I formalize the workflow of historical network analysis in collaboration with social historians, from the acquisition of their sources to their final visual analysis, and \rev{point out that visual analytics tools supporting this process should satisfy traceability, simplicity, and document reality principles to ease bask and forth between the different steps, provide tools easy to manipulate, and not distort the content of sources with modifications and simplifications.
    Particularly, I propose to model historical sources into bipartite multivariate dynamic social networks with roles to satisfy those properties.}
%Towards this goal, I first formalize the workflow of historical network analysis in collaboration with social historians, from the acquisition of their sources to their final visual analysis, and propose to model historical sources into bipartite multivariate dynamic social networks with roles to satisfy traceability, simplicity, and document reality properties.
    This modeling allows a concrete representation of historical documents, hence letting users encode, correct, and analyze their data with the same abstraction and tools.
    Leveraging this data model, I propose two interactive visual interfaces to manipulate, explore, and analyze this type of data with a focus on usability for social historians.
    First, I present ComBiNet, which allows an interactive exploration leveraging the structure, time, localization, and attributes of the data model with the help of coordinated views, a visual query system, and comparison mechanisms.
    Finding specific patterns easily and comparing them, social historians are able to find inconsistencies in their annotations and answer their high-level questions.
    The second system, PK-Clustering, is a concrete proposition to increase the usability and effectiveness of clustering mechanisms in social network visual analytics systems. It consists in a mixed-initiative clustering interface that let social scientists create meaningful clusters with the help of their prior knowledge, algorithmic consensus, and interactive exploration of the network.
    Both systems have been designed with continuous feedback from social historians, and aim to increase the traceability, simplicity, and document reality of
    \rev{visual analytics supported historical social network research.}
    I conclude with discussions on the potential merging of both systems and more globally on research directions towards better integration of visual analytics systems on the whole workflow of social historians.
%Such systems with a focus on usability can lower the requirements for the use of quantitative methods for historians and social scientists, which has always been a controversial discussion among practitioners.
    Such systems with a focus on \rev{those properties---traceability, simplicity, and document reality---can limit the introduction of bias} while lowering the requirements for the use of quantitative methods for historians and social scientists which has always been a controversial discussion among practitioners.
}

\newcommand{\abstractengtt}{This thesis aims at identifying how Visual Analytics can support historians in their social network analysis process, from the collection of historical documents to the formulation of high-level socio-historical conclusions. Historical Social Network Analysis is a method to study social relationships between groups of actors (families, institutions, companies, etc.) to understand their underlying structure while characterizing specific behaviors. Social historians are able to reconstruct relationships of the past using historical documents' content, such as marriage acts, migration forms, birth certificates, and censuses. Through visualization and analytical methods, they can describe the global structure of studied groups and explain individual behaviors through local network patterns. However, the inspection, encoding, correction, and modeling process of the historical documents leading to a finalized network is intricate and often results in inconsistencies, errors, distortions, simplifications, and traceability issues.For these reasons, social historians are not always able to make thorough historical conclusions with current analytical and visualization tools. I aim in this thesis to identify how visual analytics---the integration of data mining capabilities into visual interfaces with interaction---can support social historians in their process, from the collection of their data to the answer to high-level historical questions. Towards this goal, I formalize the workflow of historical network analysis in collaboration with social historians, from the acquisition of their sources to their final visual analysis, and point out that visual analytics tools supporting this process should satisfy traceability, simplicity, and document reality principles to ease bask and forth between the different steps, provide tools easy to manipulate, and not distort the content of sources with modifications and simplifications.Particularly, I propose to model historical sources into bipartite multivariate dynamic social networks with roles to satisfy those properties. This modeling allows a concrete representation of historical documents, hence letting users encode, correct, and analyze their data with the same abstraction and tools. Leveraging this data model, I propose two interactive visual interfaces to manipulate, explore, and analyze this type of data with a focus on usability for social historians. First, I present ComBiNet, which allows an interactive exploration leveraging the structure, time, localization, and attributes of the data model with the help of coordinated views, a visual query system, and comparison mechanisms. Finding specific patterns easily and comparing them, social historians are able to find inconsistencies in their annotations and answer their high-level questions. The second system, PK-Clustering, is a concrete proposition to increase the usability and effectiveness of clustering mechanisms in social network visual analytics systems. It consists in a mixed-initiative clustering interface that lets social scientists create meaningful clusters with the help of their prior knowledge, algorithmic consensus, and interactive exploration of the network. Both systems have been designed with continuous feedback from social historians, and aim to increase the traceability, simplicity, and document reality of visual analytics supported historical social network research. I conclude with discussions on the potential merging of both systems and more globally on research directions towards better integration of visual analytics systems on the whole workflow of social historians. Such systems with a focus on those properties---traceability, simplicity, and document reality---can limit the introduction of bias while lowering the requirements for the use of quantitative methods for historians and social scientists, which has always been a controversial discussion among practitioners.
}

\newcommand{\abstractfr}{
    Cette thèse vise à identifier comment l'analyse visuelle peut supporter les historiens dans leur processus d'analyse de réseaux sociaux, de la collecte de documents historiques jusqu'à la formulation de conclusions socio-historiques.
    L'analyse de réseaux sociaux historiques est une méthode permettant d'étudier les relations sociales au sein de groupes d'acteurs (familles, institutions, entreprises, etc.) pour comprendre leurs structures sous-jacentes tout en décrivant des comportements spécifiques.
    Les chercheurs en histoire sociale reconstruisent les relations du passé à partir du contenu de documents historiques, tel que des actes de mariage, formulaires de migration, ou des recensements.
    Utilisant des méthodes analytiques et de visualisation, les historiens peuvent décrire la structure de ces groupes et expliquer des comportements individuels à partir de motifs locaux.
    Cependant, l'inspection, l'encodage et la modélisation des sources pour obtenir un réseau finalisé provoquent souvent des erreurs, distorsions et des problèmes de traçabilité.
    Pour ces raisons, ainsi que des problèmes d'utilisabilité, les historiens ne sont pas toujours en position de faire des conclusions approfondies sur leur réseau à partir des systèmes de visualisation actuels.
    Je vise dans cette thèse à identifier comment l'analyse visuelle (la combinaison d'algorithmes statistiques intégrés à des interfaces graphiques à l'aide d'interaction) peut supporter les historiens dans leur processus, de la collecte des données jusqu'à l'analyse finale.
    Vers ce but, je formalise le processus d'une analyse de réseau historique en partant de collaborations avec des historiens, de l'acquisition des sources jusqu'à l'analyse visuelle, et pointe que les outils supportant ce processus devraient satisfaire des principes de traçabilité, simplicité et de réalité documentaire pour faciliter les va-et-vient entre les différentes étapes, avoir des outils faciles à utiliser, et à ne pas distordre le contenu des sources.
    Particulièrement, je propose de modéliser les sources historiques en réseaux sociaux bipartis multivariés dynamiques avec rôles pour satisfaire ces propriétés.
    Ce modèle représente concrètement les documents historiques, permettant aux utilisateurs d'encoder, corriger et analyser leurs données avec le même modèle et les mêmes outils.
    Je propose deux interfaces d'analyse visuelle pour manipuler, explorer et analyser ce type de données, avec un appui sur les principes de traçabilité, simplicité, et réalité documentaire.
    Je présente d'abord ComBiNet, qui permet une exploration visuelle à partir de la topologie, dynamique, localisation et attributs du réseau à l'aide de vues coordonnées, un système de requêtes visuelles, et de comparaisons.
    En trouvant des motifs facilement et en les comparant, les historiens peuvent trouver des erreurs dans leurs annotations tout en répondant à des questions historiques.
    Le second système, PK-Clustering, constitue une proposition concrète pour améliorer l'utilisabilité et l'efficacité des mécanismes de clustering dans les systèmes de visualisation de réseaux sociaux.
    L'interface permet de créer des regroupements pertinent à partir de la connaissance à priori, le consensus algorithmique et l'exploration du réseau dans un cadre d'initiative mixte.
    Les deux systèmes ont été conçu à partir des besoins et de retours continus d'historiens, et visent à augmenter la traçabilité, simplicité, et la vision réelle des sources dans l'analyse de réseaux historiques.
    Je conclus sur des discussions sur la fusion des deux systèmes et plus globalement sur la convergence vers une meilleure intégration des outils d'analyse visuelle sur le processus global des historiens.
    De tels systèmes avec une attention les propriétés de traçabilité, simplicité, et réalité documentaire peuvent limiter l'introduction de biais et abaisser les exigences pour l'utilisation de méthodes quantitatives, qui a toujours été une discussion controversée en Histoire.
}


% OLD
%\newcommand{\abstractfr}{
%    Cette thèse vise à identifier comment l'analyse visuelle peut supporter les historiens dans leur processus d'analyse de réseaux sociaux, de la collection de documents historiques jusqu'à la formulation de conclusions socio-historiques de haut niveau.
%    L'analyse de réseaux sociaux historiques est une méthode permettant d'étudier les relations sociales au sein et entre des groupes d'acteurs (familles, institutions, entreprises, etc.) pour comprendre leur structure sous-jacente tout en caractérisant des comportements spécifiques.
%    Les chercheurs en histoire sociale reconstruisent les relations du passé à partir du contenu de documents historiques, tel que des actes de marriage, formulaires de migration, ou des certificats de naissance.
%    Utilisant des méthodes analytiques et de visualisation, les historiens peuvent décrire la structure de ces groupes et expliquer des comportements individuels à partir de motifs locaux.
%    Cependant, le l'inspection, encodage, et modélisation des documents historiques pour obtenir un réseau finalisé est compliqué et provoque souvent des errerus, distortions et des problèmes de traçabilité.
%
%    De plus, des problèmes d'utilisabilité et d'interpretation analytique limitent souvent l'utilisation d'interfaces visuelles en histoire.
%    Pour ces raisons, les chercheurs en histoires sociales ne sont pas toujours en position de faire des conclusions approfondis sur leur réseau à partir des systèmes de visualisation existants.
%    Dans cette thèse, je vise à identifier comment l'analyse visuelle ---combination d'algorithmes statistiques intégré dans des interfaces visuelles à l'aide d'interaction et de manipulation directe--- peut supporter les chercheurs histooire sociale dans leur processus, de la collection aux données jusqu'à la conclusions de questions historiques de haut niveau.
%    Vers ce but, je commence par formaliser le processus d'une analyse de réseau en histoire en partant de collaborations avec des historiens, de l'acquisition des sources jusqu'à l'analyse visuelle, et pointe que les outils supportant ce processus devraient satisfaire des principes de traçabilité, simplicité et de réalité documentaire pour faciliter les vas-et-viens entre les différentes étapes, avoir des outils faciles à utiliser, et ne pas modifier le contenu des sources.
%    Particulièrement, je propose de modéliser les sources historiques en réseaux sociaux bipartis multivariés dynamiques avec roles pour satisfaire des propriétés de traçabilité, simplicité et de réalité documentaire.
%    Ce modèle permet une représentation concrete de documents historiques, permettant aux utilisateurs d'encoder, corriger et analyser leurs données avec la même abstraction et les mêmes outils.
%    Je propose deux interfacess d'analyse visuelle pour manipuler, explorer et analyser ce type de données, avec un appui sur les principes de traçabilité, simplicité, et réalité documentaire.
%    Je présente d'abord ComBiNet, qui permet une exploration visuelle à partir de la topologie, dynamique, localisation, et attributs du modèle de données à l'aide de vues coordonnées, un système de requếtes visuelles et des interactions de comparaison.
%    En trouvant des motifs facilement et en les comparant, les chercheurs en histoire sociale peuven trouver des erreurs dans leurs annotations tout en répondant à des questions historiques.
%    Le second système, PK-Clustering, constitue une proposition concrète pour améliorer l'utilisabilité et l'efficacité des mécanismes de clustering dans les systèmes de visualisation de résaux sociaux.
%    Il s'agit d'une interface de clustering d'initiative mixte permettant aux utilisateurs de créer des clusters pertinent à partir de leur connaissance à priori, le consensus algorithmique, et l'exploration du réseau.
%    Les deux systèmes ont été designés à partir de besoins et de retours continus de chercheurs en histoire sociale, et visent à augmenter la tracabilité, simplicité, et une vision réelle des documents dans le processus d'analyse de réseau historique.
%    Je conclus sur des discussions sur la fusion potentielle des ces deux système et plus globallement sur comment converger vers une meilleur intégration des outils d'analyse visuelle sur le processus global des historiens.
%    De tels systèmes avec une concentration sur ces propriétés ---traçabilité, simplicité, et réalité documentaire--- peuvent limiter l'introduction de biais et abaisser les exigences pour l'utilisation de méthodes quantitatives en histoire sociale, qui a toujours été une discussion controversée au sein de ce champs de recherche.
%}