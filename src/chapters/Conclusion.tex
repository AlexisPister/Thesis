\chapter{Conclusion}




\section{Summary}

In this thesis, we tried to give answers and leads to the high-level question of how VA can help historians following HSNA, in their entire process and not just focus on the analysis.
For this goal, we first defined the HSNA process from data acquisition to visual analysis, to define recurring pitfalls we encountered with collaborations with social scientists.
We divided the process into five steps: textual sources acquisition, digitization, annotation, network creation and network visualization/analysis, and identified recurring pitfalls for each step, such as wrongly chosen network models or named entity recognition errors.
We concluded that reality, traceability and simplicity properties should be satisfied during the overall workflow as much as possible, to respectively not introduce bias and distortion in the analysis, ease the back and forth between the analysis and the processing steps and assure reproducibility of the results, and have expressive representations and tools which are simple enough to manipulate for social scientists.
Specifically, we answer our question \qone on how to model historical documents by proposing to use \modelplural which satisfy these three conditions.
Leveraging this model, we first tried to find the right representations and types of interactions which could help social scientists answer their complex questions (\qtwo).
For this, we developed ComBiNet using feedbacks of historians, leveraging bipartite node-link representation and maps to let social scientists explore there data modeled as \modelplural, and implementing visual queries and comparisons capabilities to let them answer their potential complex questions.
Finally, we proposed PK-Clusering, a new method for clustering based on social historians needs in control over algorithmic results, as a demonstration of a VA system with the right balance of usability, control and traceability.
These two systems demonstrate that VA systems can help social historians in their overall workflow, and increase the traceability and control of the process while leveraging complex representations and algorithmic power.


\section{Discussion}\label{sec:discussion}

We discuss in this section different limitations of our work:

\noindent\textbf{Temporality.} The time is a key information for historians, as they want to contextualize the phenomena they study in a period, relative to other events.
This is why we encode time in our suggested model of \modelplural, so historians can explore and analyze this dimension of their data.
However, dynamic graphs are complex to visualize and analyze.
In our proposed interfaces ComBiNet and PK-Clustering, time is not a central part of the interactions, and historians could therefore miss some potential interesting patterns related to it.
In ComBiNet, time is encoded as an attribute in documents nodes, and users can therefore apply filters on it, and see time distributions related to the overall network, specifics documents, and filtered groups.
It allows them for example to compare two periods they are interested in.
However, the two layouts focus on the topology and the location first.
In PK-Clustering, social scientists can build a satisfactory partition based on their prior knowledge and consensus of clustering algorithms.
The prototype now consider only static clustering, which can be seen as a simplification of the real world groups which are often evolving with time.
Indeed, persons often can change groups with time and clusters can sometimes merge, split, and disappear according time.
Pk-Clustering is already a complex process for static clustering, but could be extended to the building of dynamic groups with the use od prior-knowledge time-dependant and dynamic graph clustering algorithms.
%However, clustering can be considered dynamic with evolutions in the groups, with persons changing groups, and merging and splitting of groups.
%Dynamic clusters are however harder to visualize, comprehend and manipulate.
%Similarly, expressing prior knowledge relative to time is more complicated for users.

\todo{Put figure of dynamic layout prototype}



\noindent\textbf{VA for the HSNA workflow.} Our key point in this thesis is to show that VA should be used in the overall HSNA workflow of historians.
VA could be used to help them from data collection to their final analysis in the same environment, to ease back and forth between the steps, allowing easier exploration of different analysis goals, and better traceability/reproducibility for the overall analysis.
By modeling historical documents into \modelplural (see \autoref{ch:hsna-process-and-network-modeling}), we represent the documents and their content as a network, allowing a traceability between the network entities and the original documents.
If historians find errors in the network, they can rapidly trace it back from which document the errors come from, and correct it either directly in the visual interface, or in their annotation software using the unique identifier of the document.
This modeling choice is a first step towards a better integration of the different steps into the same VA loop.
Moreover, with \name, social scientists can apply filters to study specific visions of the network and follow multiple analysis paths on different dimensions of the data.
\name therefore allow a better integration of the cleaning of the annotation, modeling and analysis/visualization steps, using the same interface.
However, it does not allow complex network transformations (such as creating simple unipartite networks) nor adding new annotations in the documents texts.
Historians still need to use ad-hoc methods for data collection and annotation, and may want to make other network transformations for specific analysis goals.



\noindent\textbf{HSNA and Social History.} HSNA is now a widely used method in quantitative history to study relational phenomena of the past, and our reflexions and tools descibred in this thesis aim at improving the workflow of historians following such a method.
Yet, historians usually have heterogeneous and various documents when they are researching an area and era of interest, and usually apply different methods at the same time to make their historical conclusions.
The core of their work consists in extracting knowledge from rigorous inspection and cross-referencing of their documents.
If providing VA tools for their HSNA analysis from start to finish is useful to them, other types of analysis methods should also be implemented in their work environments to allow them a larger set of options to make their conclusions.
This includes methods like text analysis, correlation computations, and statistical testing \cite{lemercierQuantitativeMethodsHumanities2019}.

History is also often considered a qualitative process, meaning that historians often make conclusions and hypothesis based on the reading of other sources and the qualitative analysis of their documents.
VA tools which aim to encompass the whole historic worlflow should be able to manage this type of analysis, for example by managing textual annotation management on the digital documents, similar to Jigsaw's feature for intelligence analysis \cite{staskoJigsawSupportingInvestigative2008}.
Some quantitative methods can also let users express some of their qualitative knowledge to influence the results.
For example, bayesian statistics and semi-supervised machine learning methods are based on expressing prior-knowledge which will influence the computation and results.
With PK-Clustering, historians can also express their prior-knowledge and use it as a start to find meaningful clusters, by seeing how the diversity of algorithms match their vision of the data.
VA tools for history should therefore let users follow both qualitative and quantitative inspection of their documents from data collection to final analysis, with combinations of several tools and prior-knowledge expression.

%Social historians also have to integrate their knowledge of the subject they are studying into their work, as they often have prior knowledge coming from other historical work or other sources.
%Quantitative methods should take
%Social history
%VA tools should allow historians to express this prior knowledge if possible, as PK-Clustering allow for expressing prior-knowledge as partial clusters.



\noindent\textbf{Diversity of Historical Documents.} We elaborated our reflexions on the HSNA workflow and VA tools in collaborations with historians who base their work on semi-structured documents such as marriage acts, birth certificate, and migrations forms.
These types of documents have a repetitive structure and mention people in a restricted number of relationships (spouses and witness for marriages, parents and child for birth certificate, etc.) that can be encoded as roles in a consistent manner.
Historians often leverage those types of documents in their work, as they can find them in national archives.
However, other types of textual documents can be used as historical sources, which can be less structured or without any predefined structure at all.
One example is correspondence letters, which is a type of document often studied in history \todo{cite}.
The content of letters is more verbose and vary from one to another, making the process of definying a set of relationships to encode more difficult.
\modelplural would therefore not necessarily be an efficient model to encode this type of data, and other network models may be a better fit.
Other types of quantitative methods can also be used by historians, such as text analysis.





\section{Perspectives}

We list in this section how this work could be extended, and interesting research directions for social history VA applications.


\noindent\textbf{Dynamic Layouts and Clustering.} As discussed in \autoref{sec:discussion},



\noindent\textbf{Machine Learning, Automation, and Agency.} A lot of work has been done in the recent years on machine learning, due to its rapid progress in various tasks such as questions answering, automatic driving, fraud detection, or node classification.
Machine learning has also been applied to social sciences and DH, for example for historical documents digitization \cite{philipsHistoricalDocumentProcessing2020}, or link prediction \cite{michalskiPredictingSocialNetwork2012}.
Machine learning can give state-of-the-art accuracy on many of those tasks, but often set issues on the explainability and reliability of the results in real world applications.
Several methods and approaches now focus on trying to explain the results of those black-box algorithms to the end-user.
Similarly, research is done on how to design interactive systems which leverage machine learning algorithms to guide and advise users, who still have to take the main decisions \cite{heerAgencyAutomationDesigning2019}.
PK-clusterering is based on this idea that machine learning should help users make decision based on automatic computations while letting them at the center of the analysis loop.
\name could also be extended with machine learning features and with the same agency idea, for example to suggest social scientists recurring subgraphs in the data, that could be interesting to them.
The over represented subgraphs could be a query start that the users could refine.

This idea of empowering users with the help of machine learning algorithms could be extended to the overall workflow of social historians.
In their workflow, as we saw in \autoref{ch:hsna-process-and-network-modeling} they have to manually do various tasks like transcription, Named Entity Recognition, and Named Entity Disambiguation that machine learning is efficient at.
VA interfaces could help social scientists do these tasks more easily by providing help and suggestions from  machine learning results and interactions.


\noindent\textbf{A common workflow interface.} Currently, most social scientists have to use a lot of different pieces of software, files and ad-hoc processes to follow quantitative analyses.
We provided two VA interfaces to help historians analyze their data and ease back and forth between the different steps of their analysis.
However, historians still have to collect, annotate, and process their data manually with ad-hoc methods, and may have to convert their data to various formats when using several visual analysis softwares.
All these operations make their process tedious, and usually break the traceability and reproducibility of their analyses.
In the contrary, if all the processes they do is integrated in the same visual environment, it would help the flow of their analysis, increase the traceability of the results and actions, and allow them to take several explorations paths more easily.
An interesting research direction would be to develop such systems, allowing social historians to collect, annotate, apply transform, analyze and visualize their data in the same environment, and with visual analytics capabilities.



\section{Conclusion}

