\chapter*{Aknowledgments}
%TODO: Finish

Even if those three years went incredibly fast, I feel like I grew tremendously in maturity and research skills, thanks to the many persons I encountered during this period of my life.

I would first like to thank my two advisors Jean-Daniel Fekete and Christophe Prieur, for guiding me in those three years as mentors.
First Jean-Daniel, who always knew how to develop my curiosity through interesting discussions and advices. His continuous and generous support tremendously helped me navigate those three years with excitement and joy, even during moments of doubts and unproductivity.
I will always be proud of having been his student.
Then Christophe, for his continuous welcomed feedback. I started discovering the world of social sciences through him, and he always knew how to guide my work when I was going in too complex and abstract directions.
For this thank you.

I would also like thank Guy Melançon and Ulrik Brandes to have accepted to review my manuscript on their time.
Then, Wendy Mackay, Laurent Beauguitte, and Ura Hinrichs to have accepted being part of my jury.

This thesis would not exist without the many collaborations, discussions, and exchange I had with several social scientists and historians.
Particularly, I thank Pascal Cristofoli and Nicole Dufournaud for their tremendous help, feedback, and support; this thesis would not have been possible without you.
I would also like to thank Dana Diminescu and Zacarias Moutoukias who
Finally, I would like to thank Claire Lemercier for her precious feedback and references she shared with me.

My thoughts also go to all the people of AVIZ I shared my last three years with. It was always a pleasure to come work at the lab in such a great atmoshpere.
I would first like to thank Natkamon Tovanich, for the pleasure of complaining about research life, Gaelle Richer, for all the great discussions, and Paola Valdivia, for your helpful guidance.
It was fun to navigate (and get lost a couple times) in the VAST challenge with you, and I feel I learned a lot by working on your side.
I also want to thank Mickael Sereno, for his sharing of technical (and sports!) knowledge, and Alaul Islam, for his good mood and kindness. It was a pleasure to be your office neighbors for three years, and laughing  and complaining with you about administrative issues always constituted agreable work breaks.

Finally, thanks to Catherine Plaisant, who always gave me great and appreciated feedback on my work, and Paolo Buono for his welcomed help on servers and deployments---two things I spent way more time than I thought I would.

I also want to thank all of my friends from Paris, who allowed me to do and think about other things than work: Kaelan, Paul, Bastien, Pierre, Aurelie, Jurgen
It was fun to navigate through this thesis experience together
I feel grateful to have you in my life.

My thoughts also go to my parents, who conveyed to me their open-mindness and curiosity.

In last, I thank Julia who always helped and supported me in those three years, even in the most difficult times.
I am grateful to have you in my life, as I feel our bond is stronger than ever.
I am looking forward to the next adventures which await us.










\newpage
\section*{On the usage of the pronouns we and I}

Most of the research described in this thesis was highly collaborative.
I would like to thank deeply all my collaborators for their help, support, and thoughtful discussions.
In the writing, I hence use ``we'' for collaborative parts and ``I'' for the parts I have mostly done myself.
%To highlight the collaborative part of theis work, I use ``we'' for collaborative aspects and ``I'' for the parts I have done myself.




%The core research projects in this thesis were highly collaborative. Upon reflection, I feel both lucky and grateful for my collaborator’s contributions, efforts and support throughout the thesis. I have learned a lot from them and am greatly indebted to them. In recognition of the collaborative nature of this thesis, and for ease of reading, I thus use the pronoun ’we’ when describing collaborative parts of this thesis and use "I" when it is done by myself.


















