\chapter*{Aknowledgments}

Those three years went incredibly fast, and yet I feel I grew tremendously during that time both as a person and researcher, thanks to the many persons I encountered who supported, inspired, and motivated me.
%I feel like I grew tremendously in the last three years, as a person and researcher, thanks to the many persons I encountered and who supported, inspired, and shared time with me.

I would first like to thank my two advisors Jean-Daniel and Christophe, for guiding me in those three years as great mentors.
Jean-Daniel, for his continuous and generous support which tremendously helped me navigate those three years with excitement and joy, even during moments of doubt and unproductiveness.
His door was always open to me for sharing my ideas and thoughts, which always led to great humane and scientific discussions and never lacked to motivate me.
I will always be proud of having been his student.
%who always knew how to develop my curiosity through interesting discussions and advices. His
%I always felt my ideas, thoughts, and propositions were welcome,  which led to great research and human discussions
Christophe, for his continuous welcomed feedback.
I started truly discovering the world of social sciences through him, and he always knew how to guide my work when I was going in too complex and abstract directions. For this thank you.

I also thank Guy Melançon and Ulrik Brandes for accepting to review my manuscript on their time, and Wendy Mackay, Laurent Beauguitte, and Uta Hinrichs to have accepted being part of my jury.

This thesis would not exist without the collaborations, discussions, and exchanges I had with several social scientists and historians.
Particularly, I want to thank Pascal Cristofoli and Nicole Dufournaud for their tremendous generosity, help, feedback, and support; this thesis would not have been possible without you.
I would also like to thank Dana Diminescu and Zacarias Moutoukias for the insightful discussions and the data they shared with me.
Finally, I would like to thank Claire Lemercier for her precious feedback and the useful references she pointed out to me.


I thank as well all the people of the Aviz team with whom I shared my last three years. It was always a pleasure to come work at the lab in such a nice and friendly atmosphere.
I would first like to thank Natkamon Tovanich, for all the beers and guilty pleasures of complaining about research life, Gaëlle Richer, for all the great and insightful discussions, and Paola Valdivia, for her helpful guidance.
It was fun to navigate (and get lost a couple of times) in the VAST challenge with you, and I feel I learned a lot by working on your sides.
I also want to thank Mickaël Sereno, for his sharing of technical knowledge (even though you always go a little too fast on the bike), and Alaul Islam, for his good mood and kindness. It was a pleasure to be your office neighbor for three years, as discussing, laughing, and complaining with you about administrative issues always constituted agreeable work breaks.
I also thank Jiayi who read a part of my manuscript (with Alaul and Natkamon), Sara for her inspirational productivity, and all the others who are or have been part of the team: Petra, Tobias, Lijie, Federica, Nivan, Sarkis, Marie, Xiayo.
I wish to thank as well Katia Evrat, the administrative team assistant, who always provided the needed help and guidance to go through all the paperwork and processes inherent to getting a PhD.
Finally, thanks to Catherine Plaisant, who always gave me great and appreciated feedback on my work, and Paolo Buono for his welcomed help on servers and deployments---two things I spent way more time on than I thought I would.


I also want to thank all of my friends, especially from Paris, who allowed me to escape work life when needed and spend pleasing times.
Kaelan, who always has truthful words, Paul, who always have kind words, Bastien for his mix of craziness and kindness (I should have more time to answer calls now), Pierre, whose complaining never fails to make me laughs, Aurélie, for her appreciated extroversion, Jurgen for the motivational discussions.
Hiking, playing music, going on vacation, and partying with all of you never failed to make me happy, and I feel grateful to have you in my life.
I thank as well those I do not have the chance to see that often anymore given the distance: Benjamin, as I could always count on you for support, Salma, for your great kindness, and Adrien, as your crazy life and thoughts never fail to stimulate me.


My thoughts also go to my parents, who conveyed to me their open-mindedness and curiosity.

Finally, I thank Julia, my love, who always helped and supported me in those three years, even in the most difficult times, and who even read a part of this manuscript.
I could not dream of a better partner as you continuously bring joy and laughter into my life, and never failed to cheer me up and motivate me when I was feeling down. I am grateful to have you.
After all those years I feel our bond is now stronger than ever, and I am looking forward to the next adventures that await us.










\newpage

\section*{On the usage of footnotes}

This thesis led me to read many history books and articles, in which footnotes are more than widespread. I grew rather fond of them, which explains why you will see several of them across this manuscript.


\section*{On the usage of the pronouns we and I}

Most of the research described in this thesis was highly collaborative.
I would like to thank deeply all my collaborators for their help, support, and thoughtful discussions.
In the writing, I hence use ``we'' for collaborative parts and ``I'' for the parts I have mostly done myself.
%To highlight the collaborative part of theis work, I use ``we'' for collaborative aspects and ``I'' for the parts I have done myself.




%The core research projects in this thesis were highly collaborative. Upon reflection, I feel both lucky and grateful for my collaborator’s contributions, efforts and support throughout the thesis. I have learned a lot from them and am greatly indebted to them. In recognition of the collaborative nature of this thesis, and for ease of reading, I thus use the pronoun ’we’ when describing collaborative parts of this thesis and use "I" when it is done by myself.


















