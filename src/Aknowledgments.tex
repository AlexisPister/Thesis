\chapter*{Aknowledgments}
%TODO: Finish

Those three years went incredibly fast, and yet I feel I grew tremendously in the last three years both as a person and researcher, thanks to the many persons I encountered who supported, inspired, motivated me and who shared time with me.
%I feel like I grew tremendously in the last three years, as a person and researcher, thanks to the many persons I encountered and who supported, inspired, and shared time with me.


I would first like to thank my two advisors Jean-Daniel and Christophe, for guiding me in those three years as great mentors.
Jean-Daniel, for his continuous and generous support which tremendously helped me navigate those three years with excitement and joy, even during moments of doubts and unproductivity.
His door was always open to me for sharing my ideas and thoughts, which always led to great humane and scientific discussions and never lacked to motivate me.
I will always be proud of having been his student.
%who always knew how to develop my curiosity through interesting discussions and advices. His
%I always felt my ideas, thoughts, and propositions were welcome,  which led to great research and human discussions
Christophe, for his continuous welcomed feedback.
I started truly discovering the great world of social sciences through him, and he always knew how to guide my work, especially when I was going in too complex and abstract directions. For this thank you.


I would like to thank Guy Melançon and Ulrik Brandes to have accepted to review my manuscript on their time, and Wendy Mackay, Laurent Beauguitte, and Uta Hinrichs to have accepted being part of my jury.


This thesis would not exist without the many collaborations, discussions, and exchange I had with several social scientists and historians.
Particularly, I want to thank Pascal Cristofoli and Nicole Dufournaud for their tremendous generosity, help, feedback, and support; this thesis would not have been possible without you.
I would also like to thank Dana Diminescu and Zacarias Moutoukias for sharing their ideas and data.
Finally, I would like to thank Claire Lemercier for her precious feedback and references she shared with me.


My thoughts also go to all the people of AVIZ I shared my last three years with. It was always a pleasure to come work at the lab in such a great atmoshpere.
I would first like to thank Natkamon Tovanich, for all the beers and guilty pleasures of complaining about research life, Gaelle Richer, for all the great and insightful discussions, and Paola Valdivia, for her helpful guidance.
It was fun to navigate (and get lost a couple times) in the VAST challenge with you, and I feel I learned a lot by working on your side.
I also want to thank Mickael Sereno, for his sharing of technical knowledge (even though you always go a little bit too fast in bike), and Alaul Islam, for his good mood and kindness. It was a pleasure to be your office neighbors for three years, and laughing and complaining with you about administrative issues always constituted agreable work breaks.
I also thank Jiayi who read a part of my manuscript (with Alaul and Natkamon), and all others who are or have been part of the team: Sarkis, Marie, Xiayo, Lijie, Federica, Nivan.
Finally, thanks to Catherine Plaisant, who always gave me great and appreciated feedback on my work, and Paolo Buono for his welcomed help on servers and deployments---two things I spent way more time than I thought I would.


I also want to thank all of my friends from Paris, who allowed me to do and think about other things than work: Kaelan, Paul, Bastien, Pierre, Aurelie, Jurgen
It was fun to navigate through this thesis experience together
I feel grateful to have you in my life.


My thoughts also go to my parents, who conveyed to me their open-mindness and curiosity.


In last, I thank Julia who always helped and supported me in those three years, even in the most difficult times, and who even read a part of this manuscript.
I feel grateful to have you, as you continuously bring joy and laugh in my life.
I feel our bond is stronger than ever, and I am looking forward to the next adventures that await us.










\newpage

\section*{On the usage of footnotes}

This thesis led me to read many history work, in which footnotes are more than widespread. I grew rather found of them and explain why you will see several of them across this manuscript.


\section*{On the usage of the pronouns we and I}

Most of the research described in this thesis was highly collaborative.
I would like to thank deeply all my collaborators for their help, support, and thoughtful discussions.
In the writing, I hence use ``we'' for collaborative parts and ``I'' for the parts I have mostly done myself.
%To highlight the collaborative part of theis work, I use ``we'' for collaborative aspects and ``I'' for the parts I have done myself.




%The core research projects in this thesis were highly collaborative. Upon reflection, I feel both lucky and grateful for my collaborator’s contributions, efforts and support throughout the thesis. I have learned a lot from them and am greatly indebted to them. In recognition of the collaborative nature of this thesis, and for ease of reading, I thus use the pronoun ’we’ when describing collaborative parts of this thesis and use "I" when it is done by myself.


















